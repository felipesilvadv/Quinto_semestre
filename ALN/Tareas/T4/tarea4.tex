\documentclass{article}
\usepackage[utf8]{inputenc}
\usepackage{amsmath, amsthm, amsfonts,amssymb}
\usepackage[spanish]{babel}
\usepackage{multicol}
\usepackage{listings}
\lstset{basicstyle=\footnotesize\ttfamily,breaklines=true}
\usepackage{alltt}
\usepackage{graphicx}
\usepackage{subfigure}
\usepackage{subfig}
\usepackage{float}
\usepackage{url}
\usepackage{enumerate}
\usepackage{framed}
\usepackage{color}
\usepackage{cancel}
\usepackage{wrapfig}\definecolor{shadecolor}{RGB}{250,250,250}
\usepackage{framed}
\usepackage{epstopdf}
\setlength\parindent{0pt}
\usepackage{listings}
\usepackage{color} %red, green, blue, yellow, cyan, magenta, black, white
% Operadores matemáticos y simbolos
\DeclareMathOperator{\dive}{div}
\DeclareMathOperator{\trace}{trace}
\DeclareMathOperator{\tr}{tr}
\DeclareMathOperator{\symm}{symm}
\DeclareMathOperator{\sk}{skew}
\DeclareMathOperator{\grad}{grad}
\DeclareMathOperator{\Grad}{Grad}
\DeclareMathOperator{\curl}{curl}
\DeclareMathOperator{\Curl}{Curl}
\def\R{\mbox{\(\mathbb{R}\)}}
\def\E{\mbox{\(\mathbb{E}\)}}
\def\P{\mbox{\(\mathbb{P}\)}}
\def\I{\mbox{\(\mathbb{I}\)}}
\def\L{\mbox{\(\mathbb{L}\)}}
\def\dx{\mbox{\(\,\mathrm{d}x\)}}
\usepackage{geometry}
\geometry{left=2.5cm, right=2.5cm, top=2cm, bottom=3cm}
\title{Tarea 4\\}
\author{Luis Felipe Silva De Vidts}
\begin{document}
\begin{figure}
\begin{minipage}{2.5cm}
\includegraphics[width=0.8\textwidth]{./figures/LogoUC-BN}
\end{minipage}
\begin{minipage}{14.5cm}
\vspace{4mm}
{\sc PONTIFICIA UNIVERSIDAD CAT\'OLICA DE CHILE}\\
Departamento de Matemáticas y Programa de Ingeniería Matemática y Computacional \\
{\bf IMT2111 Algebra Lineal Numérica}\\
\vspace{0mm}
\hrulefill
\end{minipage}
\end{figure}
\phantom{""}
\vspace{-5mm}
\normalsize
\begin{center}
\Huge Tarea 4\\
\normalsize Luis Felipe Silva De Vidts
\end{center}
\section*{Parte Teórica}
\subsection*{Pregunta 6}
Deduzca el Proceso de Lanczos a partir del Proceso de Arnoldi aplicado a una matriz $A$ simétrica.
\section*{Parte Práctica}
\begin{enumerate}
\item Implemente \textit{Conjugate Gradient (CG) y Conjugate Residual (CR).}\footnote{Escribo todo esto para ordenarme al hacer la tarea}\\
Ambas para resolver $Ax=b$ con $A$ SPD.
\begin{itemize}
\item He leído que, dado que CR hace dos productos matriz-vector por iteración, es común preferir CG sobre CR. Me gustaría ver algunos experimentos numéricos que afirmen o refuten esa aseveración.
\item Para que sus rutinas sean comparables en cuanto a eficiencia es necesario que ustedes realicen ambas implementaciones. Es decir, no usen librerías.
\item Comparen en diversos escenarios: Matrices con número de condición 2 alto, matrices con autovalores \textit{cluster} bien y mal condicionadas.
\item Es recomendable que usen matrices \textit{sparse} y $n\geq 1000$ (En general si la matrices son muy chicas no se observa nada de interés!).
\item En las páginas 178 (Algoritmo 6.17) y 182 (Algoritmo 6.19) del libro de Y. Saad tiene los pseudos códigos de CG y CR respectivamente.

\end{itemize}
\end{enumerate}
\subsection*{Sobre la implementación}
Cada función debe tener el siguiente encabezado
$$[x, flag, relres, iter, resvec] = METODO(A,b,tol,maxit,x_0)$$
donde los parámetros de entrada son los usuales y los de salida son 
\begin{itemize}
\item $x$: Aproximación
\item $flag$: variable que indica el estatus del método:\\
$0$ indica que el método convergió con la tolerancia especificada($tol$)\\
$1$ Alcanzó el máximo de iteraciones SIN convergencia\\
$2$ El método se estancó
\item $relres$: valor del residual relativo al final del proceso $\frac{||b-Ax||}{||b||}$
\item $iter$: número de iteraciones realizadas
\item $resvec$: vector con el residual relativo por iteración
\end{itemize}
\end{document}