\documentclass{article}
\usepackage[utf8]{inputenc}
\usepackage{amsmath, amsthm, amsfonts,amssymb}
\usepackage[spanish]{babel}
\usepackage{multicol}
\usepackage{listings}
\lstset{basicstyle=\footnotesize\ttfamily,breaklines=true}
\usepackage{alltt}
\usepackage{graphicx}
\usepackage{subfigure}
\usepackage{subfig}
\usepackage{float}
\usepackage{url}
\usepackage{enumerate}
\usepackage{framed}
\usepackage{color}
\usepackage{cancel}
\usepackage{wrapfig}\definecolor{shadecolor}{RGB}{250,250,250}
\usepackage{framed}
\usepackage{epstopdf}
\setlength\parindent{0pt}
\usepackage{listings}
\usepackage{color} %red, green, blue, yellow, cyan, magenta, black, white
% Operadores matemáticos y simbolos
\DeclareMathOperator{\dive}{div}
\DeclareMathOperator{\trace}{trace}
\DeclareMathOperator{\tr}{tr}
\DeclareMathOperator{\symm}{symm}
\DeclareMathOperator{\sk}{skew}
\DeclareMathOperator{\grad}{grad}
\DeclareMathOperator{\Grad}{Grad}
\DeclareMathOperator{\curl}{curl}
\DeclareMathOperator{\Curl}{Curl}
\def\R{\mbox{\(\mathbb{R}\)}}
\def\E{\mbox{\(\mathbb{E}\)}}
\def\P{\mbox{\(\mathbb{P}\)}}
\def\I{\mbox{\(\mathbb{I}\)}}
\def\L{\mbox{\(\mathbb{L}\)}}
\def\dx{\mbox{\(\,\mathrm{d}x\)}}
\usepackage{geometry}
\geometry{left=2.5cm, right=2.5cm, top=2cm, bottom=3cm}
\title{Tarea 3\\}
\author{Luis Felipe Silva De Vidts}
\begin{document}
\begin{figure}
\begin{minipage}{2.5cm}
\includegraphics[width=0.8\textwidth]{./figures/LogoUC-BN}
\end{minipage}
\begin{minipage}{14.5cm}
\vspace{4mm}
{\sc PONTIFICIA UNIVERSIDAD CAT\'OLICA DE CHILE}\\
Departamento de Matemáticas y Programa de Ingeniería Matemática y Computacional \\
{\bf IMT2111 Algebra Lineal Numérica}\\
\vspace{0mm}
\hrulefill
\end{minipage}
\end{figure}
\phantom{""}
\vspace{-5mm}
\normalsize
\begin{center}
\Huge Tarea 3\\
\normalsize Luis Felipe Silva De Vidts
\end{center}
\section*{Parte Teórica}
\subsection*{Pregunta 3}
Encuentre el número de condición de las siguientes matrices:\\
\begin{figure}[h!]
\begin{minipage}{4cm}
\center
$$
A_1 =  
\begin{bmatrix}
1 & a\\
a & 1
\end{bmatrix}$$
\end{minipage}
\begin{minipage}{4cm}
$$A_2 =
\begin{bmatrix}
1 & 1 + \epsilon\\
1 - \epsilon & 1
\end{bmatrix}$$
\end{minipage}
\end{figure}
\\
¿Para qué valores de $a$ y $\epsilon$ las matrices anteriores están mal condicionadas?\\

Si calculamos el número de condición en norma infinito de las matrices anteriores, tenderemos, para $A_1$:
$$A_{1}^{-1} =
\begin{bmatrix}
1 & -a\\
-a & 1
\end{bmatrix}
*\frac{1}{1-a^{2}}
$$
luego tenemos que las normas de las matrices serán:
$$||A_1||_{\infty} = 1 + a$$
$$||A_{1}^{-1}||_{\infty} = \frac{1+a}{1-a^{2}} $$
Por lo que el el condicionamiento de $A_1$ en norma infinito será:
$$K_{\infty}(A_1)=||A_1||_{\infty}*||A_{1}^{-1}||_{\infty} = \frac{(1+a)^{2}}{1-a^{2}}$$
Por lo que si $|a|\approx 1$, entonces $K_{\infty}(A_1)\approx \infty$\\

Ahora para la matriz $A_2$ tenemos que:
$$A_{2}^{-1}=
\begin{bmatrix}
1 & -1-\epsilon\\
\epsilon -1 & 1
\end{bmatrix}
$$
por lo que tenemos que las normas infinito de las matrices serán:
$$||A_2||_{\infty} = 2+\epsilon$$
$$||A_{2}^{-1}||_{\infty} = \frac{1}{\epsilon}$$
Por lo que el número de condicionamiento será:
$$K_{\infty}(A_2)=||A_2||_{\infty}*||A_{2}^{-1}||_{\infty} = \frac{2 + \epsilon}{\epsilon}$$
Entonces si $\epsilon \approx 0$, tendremos que $K_{\infty}(A_2) \approx \infty$
\subsection*{Pregunta 6}
Si dada $A \in \mathbb{R}^{n\times n}$. Muestre que si $\lambda$ es autovalor de $A^{T}A$ entonces $0\leq \lambda\leq ||A^{T}||* ||A||$.\\
% Tomar definición de valor propio, mover normas y tomar supremo para dar la desigualdad, preguntar si no debiese ser norma en vez de lambda solo.
Demuestre que si $A$ es no singular
$$k_2(A) \leq \sqrt{k_1(A)k_\infty(A)} $$

Primero mostramos la primera parte del enunciado, como $\lambda$ es valor propio de $A^{T}A$, entonces:
$$A^{T}Av = \lambda v, v\neq 0$$
Luego tomamos norma en ambos lados de la igualdad
$$||A^{T}Av|| = |\lambda|* ||v||$$
$$|\lambda|=\frac{||A^{T}Av||}{||v||}\leq \sup_{v\neq 0} \frac{||A^{T}Av||}{||v||}$$
Por definición de norma matricial nos queda
$$|\lambda|\leq \sup_{v\neq 0} \frac{||A^{T}Av||}{||v||}= ||A^{T}A||$$
Ahora por desigualdad de normas matriciales
$$|\lambda|\leq ||A^{T}A||\leq ||A^{T}||*||A||$$
con esto obtenemos la cota por arriba, pero para mostrar que $0\leq\lambda$, probaremos que $A^{T}A$ es semi-positiva definida.
Enonces tenemos que probar que se cumpla
$$ x^{T}A^{T}Ax\geq 0$$
para todo $x\neq 0$, para eso basta con tomar el vector $v = Ax$, luego tenemos que:
$$ x^{T}A^{T}Ax=v^{T}v$$
eso es equivalente a la norma de $v$ y la norma de cualquier vector es positiva o cero, $v$ puede ser cero ya que no sabemos si $A$ es de rango completo.\\
$\therefore A^{T}A$ es semi-positiva definida.\\
Entonces se cumple la desigualdad
$$0\leq \lambda\leq ||A^{T}||*||A||$$
Ahora para demostrar el segundo punto citaré unas verdades vistas en el curso de cálculo científico:
\begin{equation}
 K_1(A)\leq nK_2(A)
\end{equation}
\begin{equation}
K_2(A)\leq nK_{\infty}(A)
\end{equation}
Teniendo esas verdades es claro que se cumple
\begin{equation}
K_2(A)\leq \frac{1}{n}K_1(A)
\end{equation}
al multiplicar $(2)$ con $(3)$
nos queda 
$$K_2(A)^{2}\leq K_1(A)*K_{\infty}(A)$$
$$K_2(A)\leq \sqrt{K_1(A)*K_{\infty}(A)}$$
\subsection*{Pregunta 7}
Suponga $A \in \mathbb{R}^{n\times n}$ definida como
$$A =
\begin{bmatrix}
1&0&0&\cdots &0\\
1&1&0&\cdots &0\\
1&0&1&\cdots &0\\
\vdots&\vdots&\vdots&\ddots& 0\\
1&0&0&\cdots &1\\
\end{bmatrix}_{n\times n}
$$
Demuestre que $\lambda =1$ es autovalor de $A^{T}A$ con $x = (x_1, x_2, \cdots, x_n)\in \R^n$ autovector asociado tal que $x_1 = 0$ y $\sum_{i=2}^{n}x_i = 0$. Muestre que existen otros dos autovectores tales que $x_2 = \cdots = x_n$ y encuentre los autovalores asociados. Demuestre que:
$$ k_2(A) = \frac{1}{2}(n+1)\left(1+\sqrt{1-\frac{4}{(n+1)^{2}}}\right)$$
{\bf Note que} Para esta matriz el número de condición $2$ está relacionado con el tamaño de la matriz $A$.\\

Primero podemos ver que $\lambda=1$ es valor propio de $A^{T}A$ ya que si tomamos el vector 
$$v=
\begin{bmatrix}
0\\
\alpha_1\\
\vdots\\
\alpha_{n-1}
\end{bmatrix}
$$
también tenemos que:
\footnote{Lo calcule en Octave}
$$A^{T}A = 
\begin{bmatrix}
n&1&1&\cdots&1\\
1&1&0&\cdots&0\\
1&0&1&\cdots&0\\
\vdots&\vdots&\vdots&\ddots&\vdots\\
1&0&0&\cdots&1\\
\end{bmatrix} $$
Ahora si multiplicamos $A^{T}A$ con el vector $v$, nos quedará:
$$
\begin{bmatrix}
n&1&1&\cdots&1\\
1&1&0&\cdots&0\\
1&0&1&\cdots&0\\
\vdots&\vdots&\vdots&\ddots&\vdots\\
1&0&0&\cdots&1\\
\end{bmatrix}
\begin{bmatrix}
0\\
\alpha_1\\
\alpha_2\\
\vdots\\
\alpha_{n-1}
\end{bmatrix}
=
\begin{bmatrix}
0* n + \sum_{i=1}^{n-1}\alpha_i\\
\alpha_1\\
\alpha_2\\
\vdots\\
\alpha_{n-1}
\end{bmatrix}
=
\begin{bmatrix}
0\\
\alpha_1\\
\alpha_2\\
\vdots\\
\alpha_{n-1}
\end{bmatrix}
$$
La última igualdad se cumple porque la suma de los $\alpha_i$ es igual a $0$, por enunciado y como se cumple la igualdad anterior, $\lambda=1$ es valor propio de la matriz $A^{T}A$.\\

Ahora tenemos que buscar los vectores que cumplan con ser vectores propios de la matriz $A^{T}A$ con la condición de que todas sus componentes exceptuando la primera son iguales para eso ahora tomamos el vector:
$$u=
\begin{bmatrix}
x\\
y\\
y\\
\vdots\\
y
\end{bmatrix}
$$
luego al multiplicarlo por la matriz $A^{T}A$ nos quedará:
$$
\begin{bmatrix}
n&1&1&\cdots&1\\
1&1&0&\cdots&0\\
1&0&1&\cdots&0\\
\vdots&\vdots&\vdots&\ddots&\vdots\\
1&0&0&\cdots&1\\
\end{bmatrix}
\begin{bmatrix}
x\\
y\\
y\\
\vdots\\
y
\end{bmatrix}
=
\begin{bmatrix}
nx + (n-1)y\\
x+y\\
x+y\\
\vdots\\
x+y
\end{bmatrix}
$$
queremos que $u$ sea vector propio por lo que necesitamos que se cumplan las siguientes igualdades:
\begin{equation}
nx + (n-1)y = \lambda x
\end{equation}
\begin{equation}
x+y = \lambda y
\end{equation}
donde $\lambda$ sería el valor propio asociado, ahora despejamos $\lambda$ de ambas ecuaciones y obtenemos al igualdad:
$$\frac{nx + (n-1)y}{x} = \frac{x+y}{y}$$
que al desarrollar obtenemos la ecuación cuadrática en $x$
$$0 = x^{2}+(1-n)yx +(1-n)y^{2}$$
cuya solución para $x$ será:
$$x = \frac{-(1-n)y \pm \sqrt{(1-n)^{2}y^{2}-4(1-n)y^{2}}}{2}$$
$$x = y*\left(\frac{(n-1) \pm \sqrt{(n-1)^{2}+4(n-1)}}{2}\right)$$
Por lo que tendremos dos vectores propios a partir de esto, ya que ahora $u$ puede ser con $x$ igual a la solución con $+$ o a la solución con $-$, lo que nos dará los dos vectores propios que buscabamos, para encontrar los valores propios correspondientes basta con reemplazar en la ecuación $(5)$ y obtendremos:
$$ y*\left(\frac{(n-1) \pm \sqrt{(n-1)^{2}+4(n-1)}}{2}\right) + y = \lambda * y$$
entonces 
$$\lambda = \frac{(n-1) \pm \sqrt{(n-1)^{2}+4(n-1)}}{2} + 1$$
donde cada valor propio le corresponde a cada vector propio según la suma o resta de la raíz de la expresión.\\

Por último demostraremos la igualdad:
$$ k_2(A) = \frac{1}{2}(n+1)\left(1+\sqrt{1-\frac{4}{(n+1)^{2}}}\right)$$
Sabemos que el $K_2(A)$ es equivalente al valor del mayor valor propio de $A^{T}A$, por lo que buscaremos sus valores propios por definición y analizaremos cual es el mayor. Para eso vemos la siguiente matriz:
$$A^{T}A-\lambda I = 
\begin{bmatrix}
n-\lambda&1&1&\cdots&1\\
1&1-\lambda&0&\cdots&0\\
1&0&1-\lambda&\cdots&0\\
\vdots&\vdots&\vdots&\ddots&\vdots\\
1&0&0&\cdots&1-\lambda\\
\end{bmatrix}
$$
al calcular el determinante de esa matriz lo haremos por cofactores y lo haremos usando la primera fila para ir formando los cofactores, por lo que el primer sumando será:
$$(n-\lambda)*(1-\lambda)^{n-1}$$
luego
el segundo cofactor será:
$$ -1*\left|\begin{matrix}
1&0&0&\cdots&0\\
1&1-\lambda&0&\cdots&0\\
1&0&1-\lambda&\cdots&0\\
\vdots&\vdots&\vdots&\ddots&\vdots\\
1&0&0&\cdots&1-\lambda\\
\end{matrix}\right|
=
-1*1*(1-\lambda)^{n-2}$$
el tercer cofactor seŕa:
$$ 1*\left|\begin{matrix}
1&1-\lambda&0&\cdots&0\\
1&0&0&\cdots&0\\
1&0&1-\lambda&\cdots&0\\
\vdots&\vdots&\vdots&\ddots&\vdots\\
1&0&0&\cdots&1-\lambda\\
\end{matrix}\right|
=
1*-1*(1-\lambda)^{n-2}$$
en este caso el $-1$ es porque al tomar el determinante de la submatriz tomamos una posición impar, por lo que se multiplica por $-1$.  El resto de cofactores es de la misma forma y siempre se obtiene el mismo resultado, ya que si el $-1$ no viene dado por la posición del cofactor al principio, lo tendrá el de la posición al seguir calculando cofactores.  Por lo tanto el determinante general quedará de la forma
$$(n-\lambda)*(1-\lambda)^{n-1}-(n-1)*(1-\lambda)^{n-2}$$
Como queremos los valores propios debemos igualarlo a $0$, entonces
$$(n-\lambda)*(1-\lambda)^{n-1}-(n-1)*(1-\lambda)^{n-2}=0$$
si factorizamos por $(1-\lambda)^{n-2}$ y desarollamos llegaremos a la igualdad:
$$(1-\lambda)^{n-2}*(\lambda^{2}-(n+1)\lambda +1)=0$$
Por lo que $\lambda =1 $ es valor propio con multiplicidad $n-2$ y la solución de la ecuación cuadrática nos dará los otros dos valores propios:
$$\lambda^{2}-(n+1)\lambda +1=0$$
que tiene como soluciones:
$$\lambda = \frac{n+1\pm\sqrt{(n+1)^{2}-4}}{2}$$
Notamos que si $n\ge 2$ la raíz es mayor a cero, por lo que en el caso de la resta de la raíz estaremos obteniendo un valor menor que con la suma. (si $n=1$, caso real, $\lambda = 1$, lo que es consistente con lo anterior, ya que tiene al menos un valor propio igual a $1$). Por lo que tomaremos la solución con la suma, ya que buscamos el valor propio mayor.\\
Ahora si desarrollamos la expresión anterior, factorizando $(n+1)^{2}$ en la raíz tendremos:
$$\lambda_{\max} = \frac{(n+1)+\sqrt{(n+1)^{2}\left(1-\frac{4}{(n+1)^{2}}\right)}}{2}$$
luego
$$\lambda_{\max} = \frac{(n+1)+(n+1)\sqrt{\left(1-\frac{4}{(n+1)^{2}}\right)}}{2}$$
$$\lambda_{\max} = \frac{1}{2}(n+1)\left(1+\sqrt{1-\frac{4}{(n+1)^{2}}}\right)$$
Como $\lambda_{\max}$  es el mayor valor propio de $A^{T}A$, es equivalente a $K_2(A)$
$$\therefore K_2(A) = \frac{1}{2}(n+1)\left(1+\sqrt{1-\frac{4}{(n+1)^{2}}}\right)$$
\subsection*{Pregunta 11}
Considere el SEL $Ax=b$, donde $A \in \mathbb{R}^{n\times n}$ es una matriz dada de elementos $a_{ij}$ tales que $a_{ii} \neq 0$ para todo $i$ y $b\in \R^{n}$ es un vector dado. Considere la sucesión de vectores $\left\lbrace x^{(k)}\right\rbrace$ definida mediante el siguiente algoritmo\\
\begin{figure}[h!]
\hrule
\vspace{1mm}
{\bf Algoritmo 1 }Método iterativo
\hrule
\vspace{1mm}
Dados $x^{(0)}\in \R^{n}$, $tol \in \R^+$ y $max \in \mathbb{Z}^+$
\begin{enumerate}[1:]
\item {\bf for} $k=0,1,\dots,max$ {\bf do}
\item \hspace{5mm} $r^{(k)}=b-Ax^{(k)}$
\item \hspace{5mm} Sea $i$ tal que $|r_{i}^{(k)}|=\max_{1\leq j \leq n}|r_{j}^{(k)}|=||r^{(k)}||_{\infty}$
\item \hspace{5mm} {\bf if} $||r^{(k)}||_{\infty}\leq tol$ {\bf then}
\item \hspace{10mm} Return
\item \hspace{5mm} {\bf end if}
\item \hspace{5mm} $x_{j}^{(k+1)}=x_{j}^{(k)}$\hspace{2mm}  $\forall j\neq i$\hspace{20mm} $\triangleright$ Definición de $x^{(k+1)}$
\item \hspace{5mm} $x_{i}^{(k+1)} = x_{i}^{(k)}+\frac{r_{i}^{(k)}}{a_{ii}}$\hspace{22.5mm} $\triangleright$ Definición de $x^{(k+1)}$
\item {\bf end for}
\end{enumerate}
\vspace{1mm}
\hrule
\end{figure}
\begin{itemize}
\item Demuestre que si $a_i$ denota la i-ésima columna de $A$.
$$ r^{(k+1)} = r^{(k)}-\frac{r_{i}^{(k)}}{a_{ii}}a_i$$

Para demostrar esto tomamos el caso general de $r^{(k)}$, sabemos que
$$r^{(k+1)}= b- Ax^{(k+1)}$$
y a su vez 
$$x^{(k+1)} = 
\begin{bmatrix}
x^{(k)}_{1}\\
x^{(k)}_{2}\\
\vdots\\
x_{i}^{(k)}+\frac{r_{i}^{(k)}}{a_{ii}}\\
\vdots\\
x^{(k)}_{n}\\
\end{bmatrix}
$$
Por lo que si lo reemplazamos en la ecuación anterior y escribimos la matriz $A$ como vectores $a_i$ que serán las columnas de $A$, no nos fijamos en los valores de $x_j^{(k+1)}$ ya que estos no varian al variar $k$. Entonces nos quedará:
$$r^{(k+1)} = b -
\begin{bmatrix}
| & | & & | & &|\\
a_1 & a_2& \cdots & a_i& \cdots & a_n\\
| & | & & | & &|
\end{bmatrix}
*x^{(k+1)}
$$
si desarrollamos el producto matriz vector, nos quedará la suma de las columnas de $A$ ponderado por las componentes de $x^{(k+1)}$, ($a_i$ serán las columnas de $A$
$$r^{(k+1)}= b-\left(a_1*x_1^{(k)}+ \cdots + \left(x_{i}^{(k)}+\frac{r_{i}^{(k)}}{a_{ii}}\right)*a_i+\cdots + a_n*x_{n}^{(k)}\right)$$
distribuimos $a_i$
$$r^{(k+1)}= b-\left(a_1*x_1^{(k)}+ \cdots + x_{i}^{(k)}*a_i+\frac{r_{i}^{(k)}}{a_{ii}}*a_i+\cdots + a_n*x_{n}^{(k)}\right)$$ 
apartamos el termino de la derecha
$$r^{(k+1)}= b-\left(a_1*x_1^{(k)}+ \cdots + x_{i}^{(k)}*a_i+\cdots + a_n*x_{n}^{(k)}\right)-\frac{r_{i}^{(k)}}{a_{ii}}*a_i$$ 
con esto tenemos que lo que esta entre parentesis corresponde exactamente a $Ax^{(k)}$, entonces
$$r^{(k+1)}= b- Ax^{(k)} -\frac{r_{i}^{(k)}}{a_{ii}}*a_i$$
pero 
$$r^{(k)}=b-Ax^{(k)}$$
Por lo tanto 
$$r^{(k+1)}=r^{(k)}-\frac{r_{i}^{(k)}}{a_{ii}}*a_i$$
\item Demuestre que:
$$||r^{(k+1)}||_1 \leq \left[1-\frac{1}{n}+ \sum_{j=1,j\neq i}^{n}\left|\frac{a_{ji}}{a_{ii}}\right| \right] ||r^{(k)}||_1 $$
{\bf Ayuda:} $||x||_{\infty}\leq ||x||_1 \leq n||x||_{\infty}$ para todo $x\in \R^{n}$\\

Para esto tomamos el resultado anterior y aplicamos norma $1$ a ambos lados, entoces tenemos:
$$||r^{(k+1)}||_{1} \leq ||r^{(k)}||_1 -\left|\frac{r_{i}^{(k)}}{a_{ii}}\right|*||a_i||_1$$
luego por definición de norma $1$, obtenemos la norma $1$ de $a_i$
$$||r^{(k+1)}||_{1} \leq ||r^{(k)}||_1 -\left|\frac{r_{i}^{(k)}}{a_{ii}}\right|*\sum_{j=1}^{n}|a_{ji}|$$
si luego separamos el elemento de la suma cuando $j=i$ nos quedara la expresión:
$$||r^{(k+1)}||_{1} \leq ||r^{(k)}||_1-|r_{i}^{(k)}| -\left|\frac{r_{i}^{(k)}}{a_{ii}}\right|*\sum_{j=1,j\neq i}^{n}|a_{ji}|$$
luego, si tomamos la mayor componente de $r^{(k)}$ y cambiamos el signo de la sumatoria la desigualdad se mantiene
$$||r^{(k+1)}||_{1} \leq ||r^{(k)}||_1-||r^{(k)}||_{\infty} +\left|\frac{r_{i}^{(k)}}{a_{ii}}\right|*\sum_{j=1,j\neq i}^{n}|a_{ji}|$$
por ultimo si tomamos la norma $1$ de $r_{i}^{(k)}$ en vez del termino i-esimo, sigue siendo una cota superior para la norma de $r^{(k+1)}$
$$||r^{(k+1)}||_{1} \leq ||r^{(k)}||_1-||r^{(k)}||_{\infty} +||r^{(k)}||_1*\sum_{j=1,j\neq i}^{n}\left|\frac{a_{ji}}{a_{ii}}\right|$$
y por lo dicho en la ayuda
$$-||x||_{\infty} \leq -\frac{||x||_1}{n}$$
por lo tanto 
$$||r^{(k+1)}||_{1} \leq ||r^{(k)}||_1-\frac{||r^{(k)}||_{1}}{n} +||r^{(k)}||_1*\sum_{j=1,j\neq i}^{n}\left|\frac{a_{ji}}{a_{ii}}\right|$$
Finalmente, factorizamos y obtenemos
$$||r^{(k+1)}||_1 \leq \left[1-\frac{1}{n}+ \sum_{j=1,j\neq i}^{n}\left|\frac{a_{ji}}{a_{ii}}\right| \right] ||r^{(k)}||_1 $$
\item Deduzca una condición suficiente de convergencia para que la sucesión $\left\lbrace x^{(k)}\right\rbrace$ converja a la solución de $Ax = b$.\\

Lo necesario para asegurar convergencia es que el factor que acompaña a la norma $1$ del residual en el paso $k$ debe ser menor estricto que $1$, de esa forma aseguramos que el residual cada vez más se acercará a $0$\\

Lo que implica que 
$$1-\frac{1}{n}+ \sum_{j=1,j\neq i}^{n}\left|\frac{a_{ji}}{a_{ii}}\right|<1 $$
lo que es equivalente a
$$n*\sum_{j=1,j\neq i}^{n}\left|\frac{a_{ji}}{a_{ii}}\right|<1 $$
que se cumplirá, siempre que la norma infinito del vector $a_i$ sea menor o igual que 1, esto por la desigualdad de la ayuda que nos da la siguiente expresion:
$$||a_i||_{\infty}\leq n*\sum_{j=1,j\neq i}^{n}\left|\frac{a_{ji}}{a_{ii}}\right|<1 $$
también sería valido tomar que la norma infinito de la matriz $A$ sea menor que $1$, ya que de esa forma nos aseguramos que todas las columnas de $A$ cumplan con la desigualdad anterior.
\end{itemize}
\subsection*{Pregunta 15}
Suponga que quiere resolver el SEL $Cz = d$ con $C\in \mathbb{C}^{m\times m}$ y $d\in \mathbb{C}^{m}$ pero usted sólo tiene rutinas que trabajan con reales. Si $C = A + iB, d = b + ic$ donde $A,B,b$ y $c$ son reales. Muestre que la solución $z = x+iy$ está dada por la solución del SEL (real):
$$
\begin{bmatrix}
A&-B\\
B&A
\end{bmatrix}
\begin{bmatrix}
x\\
y
\end{bmatrix}
=
\begin{bmatrix}
b\\
c
\end{bmatrix}
$$
¿Cómo resolvería el SEL anterior sin armar la matriz de coeficientes de orden $m^{2}$?\\

Primero mostraremos que el sistema anterior es solución del sistema con números complejos. Partimos de $Cz=d$ y reemplazamos los valores dados
$$(A+iB)z = b + ic$$
$$Az + iBz = b + ic$$
Reemplazamos $z$ y nos queda
$$A(x+iy)+iB(x+iy) = b +ic$$
$$Ax + iAy + iBx -By = b +ic$$
Ahora separamos la parte real de la imaginaria y tenemos
$$Ax-By = b$$
$$\cancel{i}(Bx +Ay) = \cancel{i}c$$
Que es equivalente a la matriz de más arriba.\\

Para la segunda parte, lo resolvería invirtiendo la matriz por bloques y ver la expresión que resuelva a $x$ y a $y$. Entonces tenemos
$$\begin{bmatrix}
A & -B & \vrule & 1 & 0\\
B & A & \vrule & 0 & 1
\end{bmatrix}
$$
Al escalonar la matriz nos quedara
$$\begin{bmatrix}
1 & 0 & \vrule & A^{-1} & (A^{-1})^{2}B\\
0 & 1 & \vrule & -(A^{-1})^{2}B & A^{-1}
\end{bmatrix}
$$
Por lo tanto la solución al sistema estaŕa dado por la expresión
$$ x= A^{-1}b + (A^{-1})^{2}Bc$$
$$y = -(A^{-1})^{2}Bb + A^{-1}c$$
Con lo que se evita tener que armar la matriz completa.
\section*{Parte Práctica}
\subsection*{Pregunta 16}
Escriba dos rutinas para clacular para calcular $C=AB$ con $A \in \R^{m\times n}$ y $B\in \R^{n\times p}$. La primera debe calcular cada elemento de $C$ mediante productos internos de las fila de $A$ con las columnas de $B$. La segunda debe formar cada columna de $C$ mediante combinaciones lineales de las columnas de $A$. Compare sus rutinas en su computador. Es probable que necesite usar matrices grandes antes de observar alguna diferencia importante. Trate de encontrar información sobre el sistema de su computador (cache, política de manejo de la memoria) para tratar de explicar los resultados observados.\\

Programe con matrices aleatorias de Octave y obtuve un error de $0$ en para $n=100, 200$, para valores más grandes de $n$ obtuve errores cercanos a $10^{-11}$, en general el caso en que se usan las columnas de $A$, tenia un parecido mayor a lo que entregaba Octave haciendo $A*B$ directamente.\\

Es probable que el hecho de tomar la columnas evita que se aproximen valores al solo sacar y operar con los elementos uno a uno de la matriz.\\

No encontré información más detallada sobre mi computador.
\subsection*{Pregunta 19}
Repita el Ejemplo numérico de las páginas 300-301 del Libro Trefethen and Bau.
\subsection*{Pregunta 20}
{\bf Ejercicio didáctico para observar velocidad de convergencia de GC:} Considere las matrices $A_1$ y $A_2$ ambas simétricas positivo definidas de orden $n$ tales que los autovalores de $A_1$ están distribuidos uniformemente en $[0.95, 1.05]$ mientras que los de $A_2$ son $[100,200, 300, 400, 500]$ y los $n-5$ restantes están uniformemente distribuidos en $[0.95, 1.05]$.\\
Constuya los vectores $b_1$ y $b_2$ tales que los sistemas $A_1x=b_1$ y $A_2y=b_2$ tengan como solución el vector de unos.\\
\begin{itemize}
\item Ejecute CG con diferentes valores de $n$
\item Grafique el error relativo y el residual por iteración para ambos SEL.
\item Compare y comente los resultados. Use los resultados vistos en clases para justificar sus observaciones.\\

Por lo visto en clases sabemos que el método de gradientes conjugados es más rápido cuando la matriz tiene menos variación de sus valores propios, mientras más valores propios iguales, más rápido es el método.
\end{itemize}
\end{document}
