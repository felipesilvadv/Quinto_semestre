\documentclass[letter]{article}

\usepackage{MD_estilo}

\nombre{Luis Felipe Silva De Vidts} % Aqui va el nombre del alumno
\numtarea{5} % Aqui va el número de la tarea


\begin{document}
	
	\begin{pregunta}{1} % Aqui se coloca el número de la pregunta
		\section*{Pregunta 1}
		Sean $f(n)$ y $g(n)$  dos funciones de $\mathbb{N}$ a $\mathbb{R}^{+}$. Demuestre o refute las siguientes afirmaciones:
		\begin{enumerate}
		\item $f(n) \not \in \mathcal{O}(g(n)) $, entonces $g(n)\in \mathcal{O}(f(n))$.
		\item $f(n)\in \mathcal{O}(g(n))$, entonces $2^{f(n)}\in \mathcal{O}(2^{g(n)})$.
		\end{enumerate}
		
	\end{pregunta}
	
	\begin{pregunta}{2}
		\section*{Pregunta 2}
		Demuestre formalmente (usando la definición formal de la notación $\mathcal{O}$) que:
		\begin{enumerate}
		\item $(\log(n))^{k}\in \mathcal{O}(n^{\epsilon})$ para $k\geq 1$ y $\epsilon >0$
		\item $\sum_{i=1}^{n}n^{i} \in \mathcal{O}(2^{n*\log(n)})$
		\end{enumerate}
		
	\end{pregunta}

\end{document}