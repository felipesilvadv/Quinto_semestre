\documentclass[letter]{article}

\usepackage{MD_estilo}

\nombre{Luis Felipe Silva De Vidts} % Aqui va el nombre del alumno
\numtarea{4} % Aqui va el número de la tarea


\begin{document}
	
	\begin{pregunta}{1} % Aqui se coloca el número de la pregunta
		\section*{Pregunta 1}
		Para un conjunto $A$, sea $R_1$ y $R_2$ dos relaciones de equivalencia.
		\begin{enumerate}
		\item Demuestre que $R_1 \cap R_2$ es una relación de equivalencia.\\
		
		Para demostrar esto basta con probar que la intersección de:
		\begin{itemize}
		\item Dos relaciones reflejas es refleja.\\
		
		Esto será verdad si 
		$$\forall a\in A. (a,a) \in R_1 \cap R_2$$
		
		como $R_1$ es refleja y $R_2$ es refleja se cumple 
		$$\forall a\in A. [(a,a) \in R_1]$$
		$$\forall a\in A. [(a,a) \in R_2]$$
		entonces como $\forall a\in A$, $(a,a)$ pertenece	$R_1$ y a $R_2$ simultáneamente, entonces $(a,a)$ está en la intersección de ambas relaciones.\\
		En otras palabras 
		$$\forall a \in A. [(a,a)\in R_1 \wedge (a,a)\in R_2]\equiv \forall a\in A.[(a,a)\in R_1\cap R_2]$$
		\item Dos relaciones simétricas es simétrica.\\
		
		Esto es cierto si se cumple
		$$\forall a,b  \in A. [(a,b)\in R_1\cap R_2 \Leftrightarrow (b,a)\in R_1\cap R_2] $$
		Tenemos que se cumple 
		$$\forall a,b  \in A. [(a,b)\in R_1 \Leftrightarrow (b,a)\in R_1] $$
		$$\forall a,b  \in A. [(a,b)\in R_2 \Leftrightarrow (b,a)\in R_2] $$
		Entonces como todo elemento de $R_1$ cumple con lo anterior y al mismo tiempo todo elemento de $R_2$ también, los elementos en común cumplirán con ser simétricos.\\
		En otras palabras
		$$\forall a,b  \in A. [(a,b)\in R_1 \Leftrightarrow (b,a)\in R_1]\wedge [(a,b)\in R_2 \Leftrightarrow (b,a)\in R_1]$$$$\equiv$$ $$\forall a,b \in A. [(a,b)\in R_1 \cap R_2\Leftrightarrow (b,a)\in R_1\cap R_2]$$
		\item Dos relaciones transitivas es transitiva.\\
		
		Esto es equivalente a probar
		$$\forall a,b,c\in A. [(a,b)\in R_1\cap R_2 \wedge (b,c)\in R_1\cap R_2] \rightarrow (a,c)\in R_1\cap R_2$$
		Como $R_1$ y $R_2$ son transitivas, entonces:
		$$\forall a,b,c\in A. [(a,b)\in R_1\wedge (b,c)\in R_1] \rightarrow (a,c)\in R_1$$
		$$\forall a,b,c\in A. [(a,b)\in R_2 \wedge (b,c)\in R_2] \rightarrow (a,c)\in R_2$$
		Como todos los pares que pertenecen a $R_1$ y a $R_2$ cumplen con lo anterior, en particular los que están en ambos también lo cumplirán.\\
		
		En otras palabras:
		$$\forall a,b,c \in A. [(a,b)\in R_1 \wedge (a,b)\in R_2 \wedge (b,c)\in R_1 \wedge (b,c)\in R_2]\rightarrow [(a,c)\in R_1 \wedge (a,c) \in R_2]$$
		$$\equiv$$
		$$\forall a,b,c\in A. [(a,b)\in R_1\cap R_2 \wedge (b,c)\in R_1\cap R_2] \rightarrow (a,c)\in R_1\cap R_2$$
		\end{itemize}
		Como $R_1$ y $R_2$ son relaciones de equivalencia cumplen con los tres puntos anteriores, lo que implica que su intersección también lo cumple, lo que hace que $R_1\cap R_2$ sea una relación de equivalencia.
		\item Demuestre que si $R_1 \circ R_2=R_2 \circ R_1 $ , entonces $R_1 \circ R_2$ es una relación de equivalencia.\\
		
		Para esto suponemos que se cumple $R_1 \circ R_2=R_2\circ R_1$ y que $R_1 \circ R_2$ no es una relación de equivalencia.\\
		Pero dado lo primero sabemos que se cumple 
		$$\forall a,b \in A. [(a,b)\in R_1\circ R_2 \Leftrightarrow (a,b)\in R_2\circ R_1]$$
		Y en particular se cumple que para todo $a\in A$:
		$$(a,a)\in R_1\circ R_2$$
		Pero eso es una contradicción ya que si $R_1\circ R_2$ no es una relación de equivalencia, ésta no puede ser refleja.
		\end{enumerate}
		
	\end{pregunta}
	
	\begin{pregunta}{2}
		\section*{Pregunta 2}
		Para un conjunto $A$, sea $R\subseteq A\times A$ una relación (no necesariamente de equivalencia). Para todo $a\in A$, se define el conjunto:
		$$ [a]_{R}=\left\lbrace b\in A | (a,b)\in R\right\rbrace.$$
		Considere el conjunto $\mathcal{S}_R=\{[a]_R|a\in A\}$ y responda las siguientes preguntas.
		\begin{enumerate}
		\item Si $R$ es una relación refleja y $\mathcal{S}_R$ es una partición de $A$, ¿és $R$ una relación de equivalencia?\\
		Demuestre o de un contra-ejemplo.
		\item Si $R$ es una relación simétrica y $\mathcal{S}_R$ es una partición de $A$, ¿és $R$ una relación de equivalencia?\\
		Demuestre o de un contra-ejemplo.
		\end{enumerate}
		
	\end{pregunta}

\end{document}