\documentclass[letter]{article}

\usepackage{MD_estilo}

\nombre{Luis Felipe Silva De Vidts} % Aqui va el nombre del alumno
\numtarea{4} % Aqui va el número de la tarea


\begin{document}
	
	\begin{pregunta}{1} % Aqui se coloca el número de la pregunta
		\section*{Pregunta 1}
		Para un conjunto $A$, sea $R_1$ y $R_2$ dos relaciones de equivalencia.
		\begin{enumerate}
		\item Demuestre que $R_1 \cap R_2$ es una relación de equivalencia.
		\item Demuestre que si $R_1 \circ R_2=R_2 \circ R_1 $ , entonces $R_1 \circ R_2$ es una relación de equivalencia.
		\end{enumerate}
		
	\end{pregunta}
	
	\begin{pregunta}{2}
		\section*{Pregunta 2}
		Para un conjunto $A$, sea $R\subseteq A\times A$ una relación (no necesariamente de equivalencia). Para todo $a\in A$, se define el conjunto:
		$$ [a]_{R}=\left\lbrace b\in A | (a,b)\in R\right\rbrace.$$
		Considere el conjunto $\mathcal{S}_R=\{[a]_R|a\in A\}$ y responda las siguientes preguntas.
		\begin{enumerate}
		\item Si $R$ es una relación refleja y $\mathcal{S}_R$ es una partición de $A$, ¿és $R$ una relación de equivalencia?\\
		Demuestre o de un contra-ejemplo.
		\item Si $R$ es una relación simétrica y $\mathcal{S}_R$ es una partición de $A$, ¿és $R$ una relación de equivalencia?\\
		Demuestre o de un contra-ejemplo.
		\end{enumerate}
		
	\end{pregunta}

\end{document}