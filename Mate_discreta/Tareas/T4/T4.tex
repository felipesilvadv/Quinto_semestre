\documentclass[letter]{article}

\usepackage{MD_estilo}

\nombre{Luis Felipe Silva De Vidts} % Aqui va el nombre del alumno
\numtarea{4} % Aqui va el número de la tarea


\begin{document}
	
	\begin{pregunta}{1} % Aqui se coloca el número de la pregunta
		\section*{Pregunta 1}
		Para un conjunto $A$, sea $R_1$ y $R_2$ dos relaciones de equivalencia.
		\begin{enumerate}
		\item Demuestre que $R_1 \cap R_2$ es una relación de equivalencia.\\
		
		Para demostrar esto basta con probar que la intersección de:
		\begin{itemize}
		\item Dos relaciones reflejas es refleja.\\
		
		Esto será verdad si 
		$$\forall a\in A. (a,a) \in R_1 \cap R_2$$
		
		como $R_1$ es refleja y $R_2$ es refleja se cumple 
		$$\forall a\in A. [(a,a) \in R_1]$$
		$$\forall a\in A. [(a,a) \in R_2]$$
		entonces como $\forall a\in A$, $(a,a)$ pertenece	$R_1$ y a $R_2$ simultáneamente, entonces $(a,a)$ está en la intersección de ambas relaciones.\\
		En otras palabras 
		$$\forall a \in A. [(a,a)\in R_1 \wedge (a,a)\in R_2]\equiv \forall a\in A.[(a,a)\in R_1\cap R_2]$$
		\item Dos relaciones simétricas es simétrica.\\
		
		Esto es cierto si se cumple
		$$\forall a,b  \in A. [(a,b)\in R_1\cap R_2 \Leftrightarrow (b,a)\in R_1\cap R_2] $$
		Tenemos que se cumple 
		$$\forall a,b  \in A. [(a,b)\in R_1 \Leftrightarrow (b,a)\in R_1] $$
		$$\forall a,b  \in A. [(a,b)\in R_2 \Leftrightarrow (b,a)\in R_2] $$
		Entonces como todo elemento de $R_1$ cumple con lo anterior y al mismo tiempo todo elemento de $R_2$ también, los elementos en común cumplirán con ser simétricos.\\
		En otras palabras
		$$\forall a,b  \in A. [(a,b)\in R_1 \Leftrightarrow (b,a)\in R_1]\wedge [(a,b)\in R_2 \Leftrightarrow (b,a)\in R_1]$$$$\equiv$$ $$\forall a,b \in A. [(a,b)\in R_1 \cap R_2\Leftrightarrow (b,a)\in R_1\cap R_2]$$
		\item Dos relaciones transitivas es transitiva.\\
		
		Esto es equivalente a probar
		$$\forall a,b,c\in A. [(a,b)\in R_1\cap R_2 \wedge (b,c)\in R_1\cap R_2] \rightarrow (a,c)\in R_1\cap R_2$$
		Como $R_1$ y $R_2$ son transitivas, entonces:
		$$\forall a,b,c\in A. [(a,b)\in R_1\wedge (b,c)\in R_1] \rightarrow (a,c)\in R_1$$
		$$\forall a,b,c\in A. [(a,b)\in R_2 \wedge (b,c)\in R_2] \rightarrow (a,c)\in R_2$$
		Como todos los pares que pertenecen a $R_1$ y a $R_2$ cumplen con lo anterior, en particular los que están en ambos también lo cumplirán.\\
		
		En otras palabras:
		$$\forall a,b,c \in A. [(a,b)\in R_1 \wedge (a,b)\in R_2 \wedge (b,c)\in R_1 \wedge (b,c)\in R_2]\rightarrow [(a,c)\in R_1 \wedge (a,c) \in R_2]$$
		$$\equiv$$
		$$\forall a,b,c\in A. [(a,b)\in R_1\cap R_2 \wedge (b,c)\in R_1\cap R_2] \rightarrow (a,c)\in R_1\cap R_2$$
		\end{itemize}
		Como $R_1$ y $R_2$ son relaciones de equivalencia cumplen con los tres puntos anteriores, lo que implica que su intersección también lo cumple, lo que hace que $R_1\cap R_2$ sea una relación de equivalencia.
		\item Demuestre que si $R_1 \circ R_2=R_2 \circ R_1 $ , entonces $R_1 \circ R_2$ es una relación de equivalencia.\\
		
		Por definición de composición, tenemos  que para todo $a,b\in A$
		$$((a,b)\in R_1\circ R_2) \Leftrightarrow (\exists c \in A. [(a,c)\in R_1 \wedge (c,b)\in R_2])$$
		Por lo que si tomamos el caso $a=b$, quedará:
		$$((a,a)\in R_1\circ R_2) \Leftrightarrow (\exists c \in A. [(a,c)\in R_1 \wedge (c,a)\in R_2])$$
		y como $R_1$ y $R_2$ son reflejas sabemos que $c=a$ cumple con lo anterior\\
$\therefore R_1\circ R_2$ es refleja.\\
		
		Por otro lado notamos que $R_1 = R_1^{-1}$ y $R_2 = R_2^{-1}$, ya que son simétricas, y en ambos casos ($\forall i \in \{1,2\}$) se cumple que 
		$$R_i = R_i^{-1}\Leftrightarrow R_i\circ R_i = I_A$$
		Ahora notamos que $R_1\circ R_2$ es transitiva si, y solo si:
		$$(R_1\circ R_2)\circ (R_1\circ R_2)\subseteq R_1\circ R_2$$
		Por enunciado $R_1\circ R_2 =R_2\circ R_1$, entonces
		$$(R_1\circ R_2)\circ (R_2\circ R_1)\subseteq R_1\circ R_2$$
		por asociatividad de la composición
		$$R_1\circ (R_2\circ R_2)\circ R_1\subseteq R_1\circ R_2$$
		Por lo probado hace dos pasos 
		$$R_1\circ (I_A)\circ R_1\subseteq R_1\circ R_2$$
		Al componer con la identidad se obtiene la relación misma
		$$R_1\circ R_1\subseteq R_1\circ R_2$$
		Por lo demostrado anteriormente
		$$I_A \subseteq R_1\circ R_2$$
		Esto último es cierto ya que $R_1\circ R_2$ es refleja. \\
		Por lo tanto $R_1\circ R_2$ es transitiva, pero además es simétrica ya que 
		$$ (R_1\circ R_2)\circ (R_2\circ R_1)=I_A$$
		Lo que implica que 
		$$R_1\circ R_2 = (R_1\circ R_2)^{-1}$$
		Por lo tanto es simétrica.\\
		
		Finalmente como $R_1\circ R_2$ es refleja, simétrica y transitiva, entonces es una relación de equivalencia.
		
		\end{enumerate}
		
	\end{pregunta}
	
	\begin{pregunta}{2}
		\section*{Pregunta 2}
		Para un conjunto $A$, sea $R\subseteq A\times A$ una relación (no necesariamente de equivalencia). Para todo $a\in A$, se define el conjunto:
		$$ [a]_{R}=\left\lbrace b\in A | (a,b)\in R\right\rbrace.$$
		Considere el conjunto $\mathcal{S}_R=\{[a]_R|a\in A\}$ y responda las siguientes preguntas.
		\begin{enumerate}
		\item Si $R$ es una relación refleja y $\mathcal{S}_R$ es una partición de $A$, ¿és $R$ una relación de equivalencia?\\
		Demuestre o de un contra-ejemplo.\\
		
		La respuesta es que si, ya que si es refleja solo basta probar que $R$ es simétrica y transitiva.\\
		
		Para eso notamos que si $\mathcal{S}_R$ es partición entonces si $b \in [a]_R$ entonces $a \in [b]_R$ que es lo mismo que la relación es simétrica.\\
		
		Mientras que si $c \in [b]_R$ y $b\in [a]_R$ entonces $c\in [a]_R$ ya que $\mathcal{S}_R$ es partición, lo que significa que si un par de elementos están relacionados entre si entonces todos los que estén relacionados con el segundo estarán con el primero, ya que no pueden pertenecer a dos clases distintas.\\
		
		Por lo tanto tiene que ser transitiva y por tanto $R$ es una relación de equivalencia.
		\item Si $R$ es una relación simétrica y $\mathcal{S}_R$ es una partición de $A$, ¿és $R$ una relación de equivalencia?\\
		Demuestre o de un contra-ejemplo.\\
		
		Un contra-ejemplo a esto es tomar el conjunto $A = \{1,2\}$ y la relación $R = \{(a,b)| a\neq b\}$ donde $\neq$ es el distinto a de los números naturales, esta relación es simétrica ya que $\forall (a,b)\in N$.  $(a,b)\in R \leftrightarrow (b,a)\in R$ para este caso particular también se cumple ya que $1 \neq 2$ si, y solo si $2\neq 1$.\\
		
		Por otro lado el conjunto $\mathcal{S}_{R}$ es una partición ya que $[1]_R=\{2\}$ y $[2]_R=\{1\}$ la unión forma el conjunto $A$ y son claramente disjuntos. Pero esta relación no es refleja, ya que $(1,1)\notin R$.\\
		
		Por lo tanto $R$ no es una relación de equivalencia.
		\end{enumerate}
		
	\end{pregunta}

\end{document}
