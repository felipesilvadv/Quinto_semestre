\documentclass[letter]{article}

\usepackage{MD_estilo}

\nombre{Luis Felipe Silva De Vidts} % Aqui va el nombre del alumno
\numtarea{7} % Aqui va el número de la tarea


\begin{document}
	
	\begin{pregunta}{1} % Aqui se coloca el número de la pregunta
		\section*{Pregunta 1}
		Demuestre o de un contra ejemplo para las siguientes afirmaciones:
		\begin{enumerate}
		\item Si $a\equiv b$ (mód $m$) y $c\equiv d$ (mód $m$) con $a,b,c,d,m \in \mathbb{Z}$ y $m\geq 2$, entonces $(a-c)\equiv (b-d)$ (mód $m$).
		\item Si $a\equiv b$ (mód $m$) y $c\equiv d$ (mód $m$) con $a,b,c,d,m\in \mathbb{Z}$, $m\geq 2$ y $c,d\geq 0$, entonces $a^{c}\equiv b^{d}$ (mód $m$).
		\end{enumerate}
		
	\end{pregunta}
	
	\begin{pregunta}{2}
		\section*{Pregunta 2}
		Una expansión factorial de un número $n$ es una sumatoria de la forma:
		$$ n = a_{k}\cdot k!+ a_{k-1}\cdot (k-1)!+\ldots + a_{2}\cdot 2!+a_{1}\cdot 1! = \sum_{i=1}^{k}a_{i}\cdot i!$$
		tal que $a_{i} \in \mathbb{N}$, $0\leq a_{i}\leq i$ para $i = 1, \ldots, k$ y $a_{k}\neq 0$.
		\begin{enumerate}
		\item Demuestre que todo número entero $n\geq 1$ se puede escribir en alguna expansión factorial.
		\item Demuestre que todo número entero $n\geq 1$ tiene una única expansión factorial.
		\end{enumerate}
		
	\end{pregunta}

\end{document}