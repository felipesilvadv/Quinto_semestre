\documentclass[letter]{article}

\usepackage{MD_estilo}

\nombre{Luis Felipe Silva De Vidts} % Aqui va el nombre del alumno
\numtarea{7} % Aqui va el número de la tarea


\begin{document}
	
	\begin{pregunta}{1} % Aqui se coloca el número de la pregunta
		\section*{Pregunta 1}
		Demuestre o de un contra ejemplo para las siguientes afirmaciones:
		\begin{enumerate}
		\item Si $a\equiv b$ (mód $m$) y $c\equiv d$ (mód $m$) con $a,b,c,d,m \in \mathbb{Z}$ y $m\geq 2$, entonces $(a-c)\equiv (b-d)$ (mód $m$).\\
		
		Para demostrar esto probaremos que 
		$$ (a-c)\mod m = (b-d) \mod m$$
		entonces tenemos que el modulo distibuye sobre la suma, por lo tanto se cumple que 
		$$ (a-c)\mod m = (a\mod m - c\mod m)\mod m$$
		por enunciado $a\mod m = b\mod m$ y $c\mod m = d\mod m$, por lo que nos queda:
		$$(a-c)\mod m = (b\mod m - d\mod m)\mod m$$
		ahora aplicamos distributividad del modulo, entonces nos queda
		$$(a-c)\mod m = (b-d)\mod m$$
		Por lo tanto 
		$$(a-c) \equiv (b-d)\mod m$$
		\item Si $a\equiv b$ (mód $m$) y $c\equiv d$ (mód $m$) con $a,b,c,d,m\in \mathbb{Z}$, $m\geq 2$ y $c,d\geq 0$, entonces $a^{c}\equiv b^{d}$ (mód $m$).\\
		
		Para este caso tomamos el contra-ejemplo en que $m=7$, $a=2$, $b=9$, $c=4$ y $d=11$, donde se cumple
		$$ (2 \mod 7) = (9\mod 7) = 2$$
		$$ (4\mod 7) = (11\mod 7) = 4$$
		luego tenemos que 
		$$(2^{4}\mod 7) = (16 \mod 7) = 2 \neq 4=(31381059609\mod 7)=(9^{11}\mod 7) $$
		Por lo que no se cumple $a^{c}\equiv b^{d}\mod m$
		\end{enumerate}
		
	\end{pregunta}
	
	\begin{pregunta}{2}
		\section*{Pregunta 2}
		Una expansión factorial de un número $n$ es una sumatoria de la forma:
		$$ n = a_{k}\cdot k!+ a_{k-1}\cdot (k-1)!+\ldots + a_{2}\cdot 2!+a_{1}\cdot 1! = \sum_{i=1}^{k}a_{i}\cdot i!$$
		tal que $a_{i} \in \mathbb{N}$, $0\leq a_{i}\leq i$ para $i = 1, \ldots, k$ y $a_{k}\neq 0$.
		\begin{enumerate}
		\item Demuestre que todo número entero $n\geq 1$ se puede escribir en alguna expansión factorial.\\
		
		Primero tenemos que para $n=1$ se cumple que tiene una expansión factorial
		$$1 = 1\cdot 1!$$
		\subsection*{Lema}
		Antes de probar para el caso general, probaremos que $\sum_{i=1}^{k}i\cdot i!$ tiene como sucesor a $(k+1)!$\\
		
		Primero evaluamos el caso base $k=1$ se tiene
		$$2! = 1\cdot 1! + 1 = 1+1 = 2 = 2!$$
		Ahora para el caso general probamos que se cumple para $k+1$ suponiendo que se cumple para todos los valores anteriores a $k+1$
		$$(k+1)! = (k+1)\cdot k! = k\cdot k! + k!$$
		como $k< k+1$ entonces se cumple la hipotesis y podemos reescribir lo anterior como 
		$$ (k+1)!=k\cdot k! + \sum_{i=1}^{k-1}i\cdot i! + 1= \sum_{i=1}^{k}i\cdot i! + 1$$
		Por lo tanto por inducción fuerte, $(k+1)!$ es el sucesor de $\sum_{i=1}^{k}i\cdot i!$ para todo $k$ natural.\\
		\subsubsection*{Colorario}
		Un colorario de lo anterior es que 
		$$ k! > \sum_{i=1}^{k-1}i\cdot i! > \sum_{i=1}^{k-1}a_i\cdot i!$$
		con $0\leq a_i\leq i$.\\
		
		Ahora  probamos que $n+1$ tiene una expansión factorial dado que todos los valores anteriores tienen una descomposición factorial, $n < n +1$ por lo que tendrá una  expansión factorial
		$$ n + 1 = \sum_{i=1}^{k} a_{i}\cdot i! + 1$$
		notamos que el valor máximo que puede tomar $n$ es cuando $a_i = i$ para todo $i$, y sigue cumpliendo que tiene una expansion factorial, cualquier otra forma de escribir $n$ sería menor a esta forma y por hipotesis inductiva, tendría una expansión factorial. Entonces podemos cambiar la igualdad anterior y nos quedará
		$$ n+1 = \sum_{i=1}^{k}i\cdot i! + 1= (k+1)!$$
		por el lema anterior, por lo que $n+1$ tiene una expansión factorial, lo que significa que por inducción fuerte, todo número natural tiene una expansión factorial.
		\item Demuestre que todo número entero $n\geq 1$ tiene una única expansión factorial.\\
		
		Para esto suponemos que existen dos expansiones para un mismo número $n$
		$$ n = \sum_{i=1}^{k} a_{i}\cdot i!= a_k\cdot k! + \sum_{i=1}^{k-1} a_{i}\cdot i!=a_k\cdot k! + r_a$$
		$$ n =\sum_{i=1}^{k} b_{i}\cdot i!= b_k\cdot k! +\sum_{i=1}^{k-1} b_{i}\cdot i!=b_k\cdot k! + r_b$$
		Lo que es equivalente a la división con resto de $n$ en $k!$, ya que $r_a$ y $r_b$ son menores que $k!$ por el colorario anterior y $k!\leq n$, ya que $n$ tiene expansión factorial, y como sabemos esta expresión nos da valores únicos de resto y parte entera, por lo tanto $a_k = b_k$ y $r_a = r_b$, pero eso es una contradicción, ya que los supusimos distintos, por lo tanto la expansión factorial es única.
		\end{enumerate}
		
	\end{pregunta}

\end{document}