\documentclass[letter]{article}

\usepackage{MD_estilo}

\nombre{Luis Felipe Silva De Vidts} % Aqui va el nombre del alumno
\numtarea{2} % Aqui va el número de la tarea


\begin{document}
	
	\begin{pregunta}{1} % Aqui se coloca el número de la pregunta
		\begin{enumerate}
		\item Se pide explicar la formula de predicado $\alpha$ y su valor de verdad para los distintos dominios dados:
		$$\alpha := \forall x.\forall y. \neg (y \preceq x)\vee (\forall z. z\preceq x\vee \neg (z\preceq y)) $$
		Esta formula de predicado puede ser reescrita como:\\
		
		Por implicancia
		\begin{equation}
		\alpha\equiv \forall x.\forall y. [(y\preceq x)\rightarrow(\forall z. z\preceq x\vee \neg (z\preceq y))
		\end{equation}
		Por Conmutatividad
		\begin{equation}
		\alpha\equiv \forall x.\forall y. [(y\preceq x)\rightarrow(\neg (z\preceq y)\vee\forall z. z\preceq x)
		\end{equation}  
		Por implicancia
		\begin{equation}				
		\alpha\equiv \forall x.\forall y. [(y\preceq x)\rightarrow ((z\preceq y) \rightarrow (\forall z. z\preceq x))]
		\end{equation}
		Lo que es equivalente a decir que para todo $(x,y)$ pertenecientes al dominio de palabras y sub-palabras si $y$ es sub-palabra de $x$, entonces si $z$ es sub-palabra de $y$ se cumple que para todo $z$, $z$ es sub-palabra de $x$.\\
		Que siempre es verdadera ya que si tenemos $x,y$ y se cumple que $y$ es sub-palabra de $x$, si tomamos un tercer elemento que sea sub-palabra de $y$ entonces, independiente de la sub-palabra de $y$ que se tome, este será sub-palabra de $x$, esto se cumple ya que la sub-palabra de $y$ es un subconjunto de $y$ , y como $y$ es sub-palabra de $x$, la sub-palabra de $y$ es sub-palabra de $x$.\\
		
		Para el caso de que el dominio sea el de palabras y prefijos, la conclusión es la misma ya que el dominio de las palabras y sub-palabras es un conjunto que contiene al dominio de las palabras y prefijos, esto porque todo prefijo $x$ es sub-palabra de $x$.\\
		En otras palabras el caso en que el dominio sea el de palabras y prefijos es un caso particular de cuando el dominio es de palabras y sub-palabras, y como se cumple para todo $x,y$ en el dominio de sub-palabras y palabras, en particular se cumple para todo $x,y$ en el dominio de las palabras y sub-palabras.
		\item Podemos tomar $\beta$ como:
		$$\beta :=\forall x. \forall y. \forall z. ((y\preceq x)\wedge(z\preceq x))\rightarrow ((z\preceq y)\vee(y\preceq z)) $$
		Notamos que esto significa para el dominio de palabras y prefijos que para cualquier trío de elementos, si dos son prefijos de un elemento en común, entonces estos son prefijos entre ellos de alguna forma.\\
		Esto es cierto ya que al ser ambos prefijos, deben compartir el inicio hasta algún punto, lo que los hace prefijos entre si.\\
		
		Por otro lado $\beta$ no es verdadera para el dominio de palabras y sub-palabras, ya que se puede tomar dos sub-palabras que no compartan ningún elemento, por lo que no serán sub-palabras entre ellas.\\
		Por ejemplo si $x = matematicas $, $y = mat$ y $ z = ic$, ambos son sub-palabras de $x$ pero entre ellos no hay ningún elemento común, por lo que no son sub-palabras en ningún sentido.
		\end{enumerate}
	\end{pregunta}
	
	\begin{pregunta}{2}
		Aqui va la respuesta a la pregunta 2.
		
	\end{pregunta}

\end{document}