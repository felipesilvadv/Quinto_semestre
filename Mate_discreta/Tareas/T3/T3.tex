\documentclass[letter]{article}

\usepackage{MD_estilo}

\nombre{Luis Felipe Silva De Vidts} % Aqui va el nombre del alumno
\numtarea{3} % Aqui va el número de la tarea


\begin{document}
	
	\begin{pregunta}{1} % Aqui se coloca el número de la pregunta
		\section*{Pregunta 1}
		Sea $A$ un conjunto no vacío. Una relación binaria $R\subseteq A \times A$ se dice Euleriana si cada vez que $(a, b) \in R$ y $(a,c) \in R$, entonces se tiene que $(b,c) \in R$.
		\begin{enumerate}
		\item Sea $T$ una relación refleja y simétrica. Demuestre que $T$ es Euclideana si, y solo si, $T$ es transitiva.\\
		
		Dado que $T$ es simétrica, entonces para todo $(a,b) \in T$, $(b,a) \in T$, y como es refleja, entonces para todo $a\in A$ se cumple que $(a,a)\in T$.\\
		Se pide demostrar que $T$ es Euclideana si, y solo si, $T$ es transitiva.\\
		$(\Rightarrow)$\\
		Suponemos $T$ Euclideana, entonces se cumple que $\forall a,b,c \in A$ se tiene 
		$$ \left[(a,b)\in T\wedge(a,c) \in T\right]\rightarrow (b,c)\in T$$ 
		como $T$ es simétrica se tiene:
		$$ \left[(b,a)\in T\wedge(a,c) \in T\right]\rightarrow (b,c)\in T$$ 
		Por lo que $T$ es transitiva.\\
		$(\Leftarrow)$\\
		Ahora suponemos $T$ transitiva, entonces se cumple que $\forall a,b,c \in A$ se tiene 
		$$ \left[(a,b)\in T\wedge(b,c) \in T\right]\rightarrow (a,c)\in T$$ 
		como $T$ es simétrica se tiene:
		$$ \left[(b,a)\in T\wedge(b,c) \in T\right]\rightarrow (a,c)\in T$$ 
		Por lo tanto $T$ es Euclideana.
		
		\item Sea $T$ una relación refleja. Demuestre que $T$ es simétrica y transitiva si, y solo si, $T$ es Euclideana.\\
		
		Dado que en la parte anterior se demostró que si $T$ es simétrica y transitiva, $T$ es Euclideana, solo realizaremos la demostración en el otro sentido.\\
		$(\Leftarrow)$\\
		Si $T$ es Euclideana se tiene que 
		$\forall a,b,c \in A$:
		$$ \left[(a,b)\in T\wedge(a,c) \in T\right]\rightarrow (b,c)\in T$$
		Entonces también tendremos:
		$$ \left[(a,c)\in T\wedge(a,b) \in T\right]\rightarrow (c,b)\in T$$
		Por lo tanto $T$ es simétrica.\\
		Como $T$ es simétrica y Euclideana, entonces $T$ también es transitiva por lo demostrado en la pregunta anterior.
		\end{enumerate}
		
	\end{pregunta}
	
	\begin{pregunta}{2}
		\section*{Pregunta 2}
		Considere el conjunto $\mathcal{N}$ de todos los subconjuntos no-vacíos y finitos de $\mathbb{N}$. Formalmente $\mathcal{N} = \{S \subseteq \mathbb{N}| S$ es finito y $S\neq \emptyset\}$. Para todo $C\in \mathcal{N}$, se define $\min(C)$ como el mínimo en $C$ según el orden $\leq$ en $\mathbb{N}$. Se define la relación $R \subseteq \mathcal{N}\times \mathcal{N}$ tal que $(A,B)\in R$ si, y solo si, si $A\neq B$, entonces:
		$$\min\left((A\cup B)-(A \cap B)\right) \in A$$
		Es decir, $(A,B) \in R$ con $A\neq B$ si el mínimo de los elementos que no tienen en común $A$ y $B$ pertenece a $A$. Por ejemplo , $A = \{1,2,4,7,8\} $ y $B = \{1,2,6,8,10\}$ cumplen que $(A,B) \in R$ dado que $\min\left((A\cup B) - (A\cap B)\right) = \min\left(\{4,6,7,10\}\right) = 4$ y $4\in A$.
		\begin{enumerate}
		\item Demuestre que $R$ es refleja, antisimétrica y conexa.\\
		
		Para esto primero reescribiremos la definición de $R$:
		$$(A,B)\in R \leftrightarrow \left(A\neq B \rightarrow \min\left((A\cup B)-(A \cap B)\right) \in A\right)$$ 
		Notamos que en el caso de $A=B$ la implicancia se hace verdadera, y por lo tanto $(A,A)\in R$ para algún $A$, de la misma forma se cumple con $B$, por lo que para cualquier conjunto $A$, $(A,A)\in R$, por lo tanto $R$ es refleja.\\
		
		$R$ es antisimétrica por construcción, ya que dentro de de la función mínimo solo existirán elementos que sean solo de $A$ o solo de $B$, por lo que no puede pasar que $(A,B)\in R$ (o sea que el valor mínimo de los elementos distintos entre $A$ y $B$ pertenezca a $A$) y que al mismo tiempo $(B,A)\in R$ (que el valor mínimo de los elementos distintos entre $A$ y $B$ pertenezca a $B$), ya que ese mínimo solo pertenece a uno de los conjuntos, a menos que $A=B$, en ese caso si se cumple que $(A,B)\in R$ y $(B,A)\in R$, por lo enunciado en el paso anterior.\\
\pagebreak		
		Para probar que $R$ es conexa basta con tomar dos conjuntos de $\mathcal{N}$ cualquiera, como ambos son conjuntos finitos y distintos de $\emptyset$, para el caso en que sean iguales sabemos que $(A,B)\in R$ ya que probamos que $R$ es refleja, si son distintos entre ellos, tendrán un conjunto de elementos distintos con al menos un elemento, que pertenecerán a $A$ o $B$, pero no a ambos ($(A\cup B) - (A\cap B)$), como dicho conjunto está conformado por una cantidad finita de números naturales, este conjunto siempre tendrá un mínimo, que pertenecerá a $A$ o $B$, pero no a ambos, si suponemos que dicho mínimo pertenece a $A$, entonces se cumple $(A,B)\in R$ mientras que si el mínimo pertenece a $B$, entonces se cumple $(B,A)\in R$, lo que es equivalente a que:\\
		\begin{figure}[h!]
		\center
		\begin{minipage}{4cm}
		\center
		$$\forall A,B \in \mathcal{N}$$
		\end{minipage}
		\begin{minipage}{4cm}
		\center
		 $$(A,B)\in R \vee (B,A) \in R$$
		 \end{minipage}
		 \end{figure}
		 \\
		 Por lo tanto $R$ es conexa.\\
		\item Demuestre que $R$ es transitiva.\\
		
		Esto quiere decir que se cumple que para todo $A, B, C\in \mathcal{N}$:
		$$ \left[(A,B)\in R\wedge(B,C) \in R\right]\rightarrow (A,C)\in R$$
		Lo demostraremos por contradicción, suponemos que se cumple 
		$$ \left[(A,B)\in R\wedge(B,C) \in R\right]\rightarrow (A,C)\not\in R$$ 
		para todo $A,B,C$\\
		Ahora tomamos el caso $C=A$ y nos quedará:
		$$ \left[(A,B)\in R\wedge(B,A) \in R\right]\rightarrow (A,A)\not\in R$$
		pero eso es una contradicción, ya que $R$ es refleja, por lo tanto $R$ es transitiva.
		\end{enumerate}
	En otras palabra, $R$ es un orden total para el conjunto $\mathcal{N}$.
	\end{pregunta}

\end{document}