\documentclass[letter]{article}

\usepackage{MD_estilo}

\nombre{Luis Felipe Silva De Vidts} % Aqui va el nombre del alumno
\numtarea{3} % Aqui va el número de la tarea


\begin{document}
	
	\begin{pregunta}{1} % Aqui se coloca el número de la pregunta
		\section*{Pregunta 1}
		Sea $A$ un conjunto no vacío. Una relación binaria $R\subseteq A \times A$ se dice Euleriana si cada vez que $(a, b) \in R$ y $(a,c) \in R$, entonces se tiene que $(b,c) \in R$.
		\begin{enumerate}
		\item Sea $T$ una relación refleja y simétrica. Demuestre que $T$ es Euclideana si, y solo si, $T$ es transitiva.\\
		
		Para probar esto...
		\item Sea $T$ una relación refleja. Demuestre que $T$ es simétrica y transitiva si, y solo si, $T$ es1 Euclideana.\\
		
		Para esto tomamos....
		\end{enumerate}
		
	\end{pregunta}
	
	\begin{pregunta}{2}
		\section*{Pregunta 2}
		Considere el conjunto $\mathcal{N}$ de todos los subconjuntos no-vacíos y finitos de $\mathbb{N}$. Formalmente $\mathcal{N} = \{S \subseteq \mathbb{N}| S$ es finito y $S\neq \emptyset\}$. Para todo $C\in \mathcal{N}$, se define $\min(C)$ como el mínimo en $C$ según el orden $\leq$ en $\mathbb{N}$. Se define la relación $R \subseteq \mathcal{N}\times \mathcal{N}$ tal que $(A,B)\in R$ si, y solo si, si $A\neq B$, entonces:
		$$\min\left((A\cup B)-(A \cap B)\right) \in A$$
		Es decir, $(A,B) \in R$ con $A\neq B$ si el mínimo de los elementos que no tienen en común $A$ y $B$ pertenece a $A$. Por ejemplo , $A = \{1,2,4,7,8\} $ y $B = \{1,2,6,8,10\}$ cumplen que $(A,B) \in R$ dado que $\min\left((A\cup B) - (A\cap B)\right) = \min\left(\{4,6,7,10\}\right) = 4$ y $4\in A$.
		\begin{enumerate}
		\item Demuestre que $R$ es refleja, antisimétrica y conexa.\\
		
		Para demostrar esto tenemos que ....
		\item Demuestre que $R$ es transitiva.\\
		
		Para demostrar esta parte ......
		\end{enumerate}
	En otras palabra, $R$ es un orden total para el conjunto $\mathcal{N}$.
	\end{pregunta}

\end{document}